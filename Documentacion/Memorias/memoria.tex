\documentclass[a4paper,11pt,oneside]{memoir}

% Castellano
\usepackage[spanish]{babel}
\selectlanguage{spanish}
\usepackage[utf8]{inputenc}
\usepackage{placeins}
\usepackage{caption}
\usepackage{subcaption}
\usepackage[linesnumbered,ruled,vlined,spanish]{algorithm2e}


\RequirePackage{booktabs}
\RequirePackage[table]{xcolor}
\RequirePackage{xtab}
\RequirePackage{multirow}

% Links
\usepackage[colorlinks]{hyperref}
\hypersetup{
	allcolors = {red}
}

% Ecuaciones
\usepackage{amsmath}

% Rutas de fichero / paquete
\newcommand{\ruta}[1]{{\sffamily #1}}

% Párrafos
\nonzeroparskip


% Imagenes
\usepackage{graphicx}
\newcommand{\imagen}[2]{
	\begin{figure}[!h]
		\centering
		\includegraphics[width=0.9\textwidth]{#1}
		\caption{#2}\label{fig:#1}
	\end{figure}
	\FloatBarrier
}

\newcommand{\imagenflotante}[2]{
	\begin{figure}%[!h]
		\centering
		\includegraphics[width=0.9\textwidth]{#1}
		\caption{#2}\label{fig:#1}
	\end{figure}
}



% El comando \figura nos permite insertar figuras comodamente, y utilizando
% siempre el mismo formato. Los parametros son:
% 1 -> Porcentaje del ancho de página que ocupará la figura (de 0 a 1)
% 2 --> Fichero de la imagen
% 3 --> Texto a pie de imagen
% 4 --> Etiqueta (label) para referencias
% 5 --> Opciones que queramos pasarle al \includegraphics
% 6 --> Opciones de posicionamiento a pasarle a \begin{figure}
\newcommand{\figuraConPosicion}[6]{%
  \setlength{\anchoFloat}{#1\textwidth}%
  \addtolength{\anchoFloat}{-4\fboxsep}%
  \setlength{\anchoFigura}{\anchoFloat}%
  \begin{figure}[#6]
    \begin{center}%
      \Ovalbox{%
        \begin{minipage}{\anchoFloat}%
          \begin{center}%
            \includegraphics[width=\anchoFigura,#5]{#2}%
            \caption{#3}%
            \label{#4}%
          \end{center}%
        \end{minipage}
      }%
    \end{center}%
  \end{figure}%
}

%
% Comando para incluir imágenes en formato apaisado (sin marco).
\newcommand{\figuraApaisadaSinMarco}[5]{%
  \begin{figure}%
    \begin{center}%
    \includegraphics[angle=90,height=#1\textheight,#5]{#2}%
    \caption{#3}%
    \label{#4}%
    \end{center}%
  \end{figure}%
}
% Para las tablas
\newcommand{\otoprule}{\midrule [\heavyrulewidth]}
%
% Nuevo comando para tablas pequeñas (menos de una página).
\newcommand{\tablaSmall}[5]{%
 \begin{table}
  \begin{center}
   \rowcolors {2}{gray!35}{}
   \begin{tabular}{#2}
    \toprule
    #4
    \otoprule
    #5
    \bottomrule
   \end{tabular}
   \caption{#1}
   \label{tabla:#3}
  \end{center}
 \end{table}
}

%
% Nuevo comando para tablas pequeñas (menos de una página).
\newcommand{\tablaSmallSinColores}[5]{%
 \begin{table}[H]
  \begin{center}
   \begin{tabular}{#2}
    \toprule
    #4
    \otoprule
    #5
    \bottomrule
   \end{tabular}
   \caption{#1}
   \label{tabla:#3}
  \end{center}
 \end{table}
}

\newcommand{\tablaApaisadaSmall}[5]{%
\begin{landscape}
  \begin{table}
   \begin{center}
    \rowcolors {2}{gray!35}{}
    \begin{tabular}{#2}
     \toprule
     #4
     \otoprule
     #5
     \bottomrule
    \end{tabular}
    \caption{#1}
    \label{tabla:#3}
   \end{center}
  \end{table}
\end{landscape}
}

%
% Nuevo comando para tablas grandes con cabecera y filas alternas coloreadas en gris.
\newcommand{\tabla}[6]{%
  \begin{center}
    \tablefirsthead{
      \toprule
      #5
      \otoprule
    }
    \tablehead{
      \multicolumn{#3}{l}{\small\sl continúa desde la página anterior}\\
      \toprule
      #5
      \otoprule
    }
    \tabletail{
      \hline
      \multicolumn{#3}{r}{\small\sl continúa en la página siguiente}\\
    }
    \tablelasttail{
      \hline
    }
    \bottomcaption{#1}
    \rowcolors {2}{gray!35}{}
    \begin{xtabular}{#2}
      #6
      \bottomrule
    \end{xtabular}
    \label{tabla:#4}
  \end{center}
}

%
% Nuevo comando para tablas grandes con cabecera.
\newcommand{\tablaSinColores}[6]{%
  \begin{center}
    \tablefirsthead{
      \toprule
      #5
      \otoprule
    }
    \tablehead{
      \multicolumn{#3}{l}{\small\sl continúa desde la página anterior}\\
      \toprule
      #5
      \otoprule
    }
    \tabletail{
      \hline
      \multicolumn{#3}{r}{\small\sl continúa en la página siguiente}\\
    }
    \tablelasttail{
      \hline
    }
    \bottomcaption{#1}
    \begin{xtabular}{#2}
      #6
      \bottomrule
    \end{xtabular}
    \label{tabla:#4}
  \end{center}
}

%
% Nuevo comando para tablas grandes sin cabecera.
\newcommand{\tablaSinCabecera}[5]{%
  \begin{center}
    \tablefirsthead{
      \toprule
    }
    \tablehead{
      \multicolumn{#3}{l}{\small\sl continúa desde la página anterior}\\
      \hline
    }
    \tabletail{
      \hline
      \multicolumn{#3}{r}{\small\sl continúa en la página siguiente}\\
    }
    \tablelasttail{
      \hline
    }
    \bottomcaption{#1}
  \begin{xtabular}{#2}
    #5
   \bottomrule
  \end{xtabular}
  \label{tabla:#4}
  \end{center}
}



\definecolor{cgoLight}{HTML}{EEEEEE}
\definecolor{cgoExtralight}{HTML}{FFFFFF}

%
% Nuevo comando para tablas grandes sin cabecera.
\newcommand{\tablaSinCabeceraConBandas}[5]{%
  \begin{center}
    \tablefirsthead{
      \toprule
    }
    \tablehead{
      \multicolumn{#3}{l}{\small\sl continúa desde la página anterior}\\
      \hline
    }
    \tabletail{
      \hline
      \multicolumn{#3}{r}{\small\sl continúa en la página siguiente}\\
    }
    \tablelasttail{
      \hline
    }
    \bottomcaption{#1}
    \rowcolors[]{1}{cgoExtralight}{cgoLight}

  \begin{xtabular}{#2}
    #5
   \bottomrule
  \end{xtabular}
  \label{tabla:#4}
  \end{center}
}


















\graphicspath{ {./img/} }

% Capítulos
\chapterstyle{bianchi}
\newcommand{\capitulo}[2]{
	\setcounter{chapter}{#1}
	\setcounter{section}{0}
	\chapter*{#2}
	\addcontentsline{toc}{chapter}{#2}
	\markboth{#2}{#2}
}

% Apéndices
\renewcommand{\appendixname}{Apéndice}
\renewcommand*\cftappendixname{\appendixname}

\newcommand{\apendice}[1]{
	%\renewcommand{\thechapter}{A}
	\chapter{#1}
}

\renewcommand*\cftappendixname{\appendixname\ }

% Formato de portada
\makeatletter
\usepackage{xcolor}
\newcommand{\tutor}[1]{\def\@tutor{#1}}
\newcommand{\course}[1]{\def\@course{#1}}
\definecolor{cpardoBox}{HTML}{E6E6FF}
\def\maketitle{
  \null
  \thispagestyle{empty}
  % Cabecera ----------------
\noindent\includegraphics[width=\textwidth]{cabecera}\vspace{1cm}%
  \vfill
  % Título proyecto y escudo informática ----------------
  \colorbox{cpardoBox}{%
    \begin{minipage}{.8\textwidth}
      \vspace{.5cm}\Large
      \begin{center}
      \textbf{TFG del Grado en Ingeniería Informática}\vspace{.6cm}\\
      \textbf{\LARGE\@title{}}
      \end{center}
      \vspace{.2cm}
    \end{minipage}

  }%
  \hfill\begin{minipage}{.20\textwidth}
    \includegraphics[width=\textwidth]{escudoInfor}
  \end{minipage}
  \vfill
  % Datos de alumno, curso y tutores ------------------
  \begin{center}%
  {%
    \noindent\LARGE
    Presentado por \@author{}\\ 
    en Universidad de Burgos --- \@date{}\\
    Tutor: \@tutor{}\\
  }%
  \end{center}%
  \null
  \cleardoublepage
  }
\makeatother

\newcommand{\nombre}{Ismael Tobar García} %%% cambio de comando

% Datos de portada
\title{Dieta por Dientes}
\author{\nombre}
\tutor{D. Álvar Arnaiz González y Dr. José Francisco Diez Pastor}
\date{\today}

\begin{document}

\maketitle



\null\cleardoublepage


%%%%%%%%%%%%%%%%%%%%%%%%%%%%%%%%%%%%%%%%%%%%%%%%%%%%%%%%%%%%%%%%%%%%%%%%%%%%%%%%%%%%%%%%
\pagestyle{empty}


\noindent\includegraphics[width=\textwidth]{cabecera}\vspace{1cm}

\noindent Dr. José Francisco Diez Pastor y D. Álvar Arnaiz González, profesores del departamento de Departamento de Ingeniería Civil, área de Lenguajes y Sistemas Informáticos.
		

\noindent Expone:

\noindent Que el alumno D. \nombre, con DNI 71286542-C, ha realizado el Trabajo final de Grado en Ingeniería Informática titulado Dieta por Dientes. 

\noindent Y que dicho trabajo ha sido realizado por el alumno bajo la dirección del que suscribe, en virtud de lo cual se autoriza su presentación y defensa.

\begin{center} %\large
En Burgos, {\large \today}
\end{center}

\vfill\vfill\vfill

% Author and supervisor
\begin{minipage}{0.45\textwidth}
\begin{flushleft} %\large
Vº. Bº. del Tutor:\\[2cm]
D. Álvar Arnaiz González
\end{flushleft}
\end{minipage}
\hfill
\begin{minipage}{0.45\textwidth}
\begin{flushleft} %\large
Vº. Bº. del tutor:\\[2cm]
Dr. Jóse Francisco Diez Pastor
\end{flushleft}
\end{minipage}
\hfill

\vfill

% para casos con solo un tutor comentar lo anterior
% y descomentar lo siguiente
%Vº. Bº. del Tutor:\\[2cm]
%D. nombre tutor


\null\cleardoublepage
\null\cleardoublepage




\frontmatter

% Abstract en castellano
\renewcommand*\abstractname{Resumen}
\begin{abstract}

La paleontología es una ciencia natural de investigación, que se encarga del estudio de restos fósiles, para conseguir tanto una reconstrucción a nivel físico y tangible, de los restos encontrados, como una reconstrucción a nivel de comportamiento a partir de las marcas imperfecciones y otros aspectos que se determinen que no son naturales de dichos restos. Todo ello lleva la finalidad de determinar y reconstruir tanto el individuo u organismo como su comportamiento.
Este área en Burgos es de particular importancia, porque tenemos algunos yacimientos muy importantes, como los localizados en la Sierra de Atapuerca.% \cite{wiki:atapuerca}.

En la antropología, como hemos mencionado anteriormente, se estudia también el comportamiento de los individuos y animales, en este caso nos centraremos en los individuos humanos, concretamente, en el modo de vida de las poblaciones humanas.
Uno de los datos que podemos extraer del análisis de los restos humanos es su alimentación. Los distintos alimentos producen distintas erosiones en la superficie de los dientes y, el análisis de estas erosiones, permite estimar la dieta.

Los procesos repetitivos son susceptibles a fallos debido al cansancio o la falta de concentración, la tecnología permite minimizar estos fallos y automatizarlos ya que un ordenador o una maquina no se cansa y puede trabajar mas rápido que nosotros. Esto hace que nos planteemos usar los recursos hoy en día disponibles, como es la tecnología, para ayudar a avanzar y a facilitar el trabajo a los investigadores de estos campos, ya que con herramientas como esta, reducimos el tiempo de hacer trabajos repetitivos, porque reducimos la tarea que mas cuesta que es el trabajo manual.

Pero esto no quiere decir que un programa haga su función, este tipo de herramientas son una ayuda a su trabajo. Siempre hay que darle una supervisión porque no son sistemas perfectos y el experto tendrá que corregir o añadir lo que no vean dichos sistemas.

En resumen, el proyecto va a consistir en detectar las marcas que se produjeron por la ingesta de alimentos en los primeros homínidos, sobre imágenes ya pintadas por paleontólogos. Medir sus longitudes, sus ángulos y clasificarlas, en función de la propuesta de~\cite{Lalueza:perez}. También va a permitir pintar las marcas desde cero sobre las imágenes y un modo en pruebas que detectará desde cero las lineas pintada en los dientes.

\end{abstract}

\renewcommand*\abstractname{Descriptores}
\begin{abstract}
Python, Procesamiento de imágenes, aprendizaje automático, paleontología, dieta.  \ldots
\end{abstract}

\clearpage

% Abstract en inglés
\renewcommand*\abstractname{Abstract}
\begin{abstract}
Paleontology is a natural science of research, which is responsible for the study of fossil remains, to achieve, physical and tangible reconstruction of the remains found, as a reconstruction at the level of behavior from the marks imperfections and other aspects Which are determined to be unnatural to such remains. All this has the purpose of determining and rebuilding the individual or organism as well as its behavior.
This area in Burgos is of particular importance because we have some very important deposits, such as those located in the Sierra de Atapuerca.% \ Cite {wiki: atapuerca}.

In anthropology, as mentioned above, we also study the behavior of individuals and animals, in this case we will focus on human individuals, specifically the way of life of human populations.
One of the data that can be extracted from the analysis of the human remains is its feeding. The different foods produce different erosions on the surface of the teeth and, the analysis of these erosions, allows to estimate the diet.

The repetitive processes are susceptible to failures due to fatigue or lack of concentration, the technology allows to minimize these failures and automate them since a computer or a machine does not get tired and can work faster than us. This makes us consider using the resources available today, such as technology, to help advance and facilitate the work of researchers in these fields, because with tools like this, we reduce the time to do repetitive work, because we reduce The task that costs the most is manual labor.

But this does not mean that a program does its job, these kinds of tools are an aid to their work. Always have to give a supervision because they are not perfect systems and the expert will have to correct or add what they do not see such systems.

In short, the project will consist of detecting the marks that were produced by the food intake in the first hominids, on images already painted by paleontologists. Measure their lengths, their angles and classify them, according to the proposal of ~ \ cite {Lalueza: perez}. It will also allow to paint the marks from scratch on the images and a mode in tests that will detect from scratch the lines painted in the teeth.
\end{abstract}

\renewcommand*\abstractname{Keywords}
\begin{abstract}

Python, Image processing, Deep Learning, paleontology, diet.\end{abstract}

\clearpage

% Indices
\tableofcontents

\clearpage

\listoffigures

\clearpage

%\listoftables

%\clearpage

\mainmatter
\capitulo{1}{Introducción}



%\maketitle
\section{Gestores de tareas:}
\subsection{Trello}
Es una pizarra virtual también conocida como canvas en la cual podemos organizar nuestros proyectos a través de su aplicación web de forma fácil e intuitiva desde cualquier equipo en el que introduzcamos nuestra cuenta ya que al estar en la nube no se pierde nuestra información ni aunque se degrade el sistema.
\\

\textbf{Ventajas:}

\begin{itemize}
\item Es muy rápido su aprendizaje y su uso así como su simplicidad.

\item La hemos usado en el transcurso de la carrera por lo que ya estaríamos familiarizados con el entorno.
\end{itemize}

\textbf{Desventajas:}

\begin{itemize}
\item No esta integrado dentro de nuestro repositorio por lo que lo tendríamos que usar como una herramienta mas en la que al final duplicaríamos trabajo.
\end{itemize}


\subsection{Version One}
Es un gestor de tareas online en el cual podemos gestionar todas las tareas de nuestros proyectos así como realizar un seguimiento de forma visual del estado del proyecto y de las características del mismo. 
\\

\textbf{Ventajas:}

\begin{itemize}
\item Podemos incluir código por lo que es mas completo que otras herramientas de gestión de tareas.
\end{itemize}

\textbf{Desventajas:}

\begin{itemize}
\item Como antes hemos indicado en este caso también seria algo que nos duplicaría trabajo al no estar integrado en nuestro repositorio seria también una herramienta a parte que no se comunicaría con nuestro repositorio.
\end{itemize}


\subsection{zenhub}
Es una herramienta similar a Trello que también es a modo de pizarra donde ver los cambios y el estado de un vistazo pero con algunas diferencias.
\\

\textbf{Ventajas:}

\begin{itemize}
\item La ventaja principal es que podemos integrarlo desde GitHub por lo que ya no seria necesario duplicar trabajo con el uso de una aplicación externa.
\end{itemize}

\textbf{Desventajas:}

\begin{itemize}
\item No podremos añadir código pero tampoco es algo catastrófico ya que justo en el repositorio donde se integra la herramienta podemos visualizar dicho código.

\item Por este motivo he decidido usar esta herramienta.
\end{itemize}

\section{Gestores de versiones:}
\subsection{GitHub}
Es un repositorio de versiones donde el código queda organizado por tareas(issues)y las versiones cada vez que hacemos un commit se actualiza las clases de código mostrando lo que ha cambiado.
El software esta escrito en Ruby usando el framework Ruby on Rails.
\\


\textbf{Ventajas:}

\begin{itemize}
\item Es de los repositorios mas usados y esta basado en git.

\item El código es publico y cualquiera que le interese te puede proponer cambios en el mismo seguirte y ver el proyecto.

\item Las distintas versiones del código están en la nube por lo que si se nos borra el contenido del disco duro aun así podremos recuperar lo.
\end{itemize}

\textbf{Desventajas:}

\begin{itemize}
\item También puedes tener proyectos privados pero al no influir sobre este proyecto no pasa nada.

\item Hemos decidido usar este repositorio al ser uno de los mas utilizados y guardar mucha similitud con sus competidores por lo que sabiendo usar este podriamos usar los demas sin demasiados problemas.
\end{itemize}

\subsection{Bitbucket}
Este servicio es muy similar al anterior tambien esta basado en Git y ademas en Mercurial.
Este repositorio web tambien guarda nuestro codigo y nuestras iteraciones sobre el proyecto para asi tener una vision mas completa sobre nuestro trabajo.
Este software esta escrito en Python.
\\

\textbf{Ventajas:}
\begin{itemize}

\item Es un repositorio muy usado y que esta basado en git que es casi la referencia en este tipo de proyectos.

\item Tiene cuentas gratuitas para proyectos privados y publicos.

\end{itemize}

\textbf{Desventajas:}

\begin{itemize}
\item No se puede incluir mas de 5 personas en los proyectos gratuitos.
\end{itemize}




\end{document}
 
\capitulo{2}{Objetivos del proyecto}

Este apartado explica de forma precisa y concisa cuales son los objetivos que se persiguen con la realización del proyecto. Se puede distinguir entre los objetivos marcados por los requisitos del software a construir y los objetivos de carácter técnico que se plantean a la hora de llevar a la práctica el proyecto.

\section{Objetivos}
A continuación se mostrara el esquema con todos los puntos a tratar en este proyecto.
\begin{itemize}
\item Analizar el problema planteado por la doctora María Rebeca García, nuestra colaboradora del Laboratorio de Evolución Humana, y buscar una solución:
	\begin{itemize}
		\item Primero tenemos que documentarnos, y saber que son las líneas a detectar.
		\item Buscar una solución para detectar dichas estrías pintadas, de forma automática, ya que hasta ahora era un problema manual que tenían que medir a mano.
		\item Permitir que se pinten a través de la aplicación para futuras imágenes.
		\item Iniciar la forma de detección completamente automática.
	\end{itemize}
\item Crear un notebook para el procesado:
	\begin{itemize}
		\item Crear un prototipo interactivo a través de los notebooks de jupyter.
		\item Procesar la imagen para obtener la máscara o imagen binarizada con las líneas.
		\item Detectar los segmentos que se corresponden con las líneas.
		\item Juntar los segmentos para obtener líneas reales.
	\end{itemize}
\item Crear la aplicación:
	\begin{description}
		\item Crear una interfaz gráfica con la imagen y botonería.
		\item [Modo 1] Modo semi-automático, detección, conteo y análisis de estrías.
		\item [Modo 2] Permitir pintar estrías y corrección de segmentos.
		\item [Modo 3] Completamente automático.
	\end{description}
\end{itemize}

\section{Resumen}

El problema va a consistir en detectar las líneas que tienen ya pintadas y detectar desde cero las no pintadas, en las imágenes, para poder automatizar dicho proceso, ya que los pasos anteriores eran:
\begin{itemize}
\item Pintar las estrías encima de la imagen.
\item Obtener de las estrías, de forma manual, su tamaño, ángulo, dirección.
\end{itemize}
Y los pasos a través de nuestra aplicación son:
\begin{itemize}
\item Abrir la imagen y seleccionar el color de las líneas
\item Dar al botón de calcular las líneas y guardarlo: Tendíamos el CSV con los estadísticos y atributos de ellas.
\end{itemize}
Como puede intuirse, nuestros pasos nos devuelven las estrías de forma mucho mas cómoda y mas rápida que buscándolas a mano.

\subsection{Crear un prototipo} 
En esta parte hemos pensado que sería más cómodo, antes de ponernos a diseñar o implementar, comprobar que lo que tenemos pensado para resolver el problema funcione.

Crear un Notebook de Jupyter (ver sección \ref{notebook:jupiter}), no nos exige programar nada de la GUI, al ser interactivo, vamos a implementar todos los pasos necesarios para resolver el problema. 
Una vez conseguido, como producto tenemos el núcleo de cálculo y pasaremos a hacer el diseño.

\subsection{Crear la aplicación}
Llegados a este punto ya tendremos todo lo necesario para hacer nuestras clases y nuestra aplicación.

Así que ya tenemos todo lo necesario y solo queda el producto entregable que será la aplicación enlazada con la GUI junto con la creación de botones, pestañas, ventanas con sus respectivas implementaciones y funcionalidades.
\capitulo{3}{Conceptos teóricos}

\section{Espacios de color }
El color que percibimos en los objetos que nos rodean depende de la radiación reflejada en ellos. Según los estudios, nosotros como humanos tenemos un rango ``de luz visible'', ese rango son en verdad tres frecuencias diferentes dentro del rango 769THz a 384THz~\cite{Manual:HAE}.\\
Por lo que en verdad una imagen que percibimos es la unión de las tres frecuencias diferentes y para poder simular este hecho las maquinas simulan esta capacidad innata de los humanos creando los espacios de color que son modelos matemáticos para representar en una maquina lo que se observa en la figura \ref{fig:3.1}.

\begin{figure}[h]
\centering
\includegraphics[width=.9\textwidth]{EspacioDeColor}
\caption{Frecuencias de luz visible \cite{Manual:HAE}}
\label{fig:3.1}
\end{figure}

\subsection{RGB}
El modelo RGB es usado por todos los sistemas digitales para la representación y captura de imágenes.\\
Se divide en tres canales como se muesra en la figura \ref{fig:3.2}\\
\begin{itemize}
	\item R: canal del rojo (RED) contiene la intensidad de rojo de cada pixel\\
	\item G: canal del verde (GREEN) contiene la intensidad de verde de cada pixel\\
	\item B: canal del azul (BLUE) contiene la intensidad de azul de cada pixel\\ 
\end{itemize}

\begin{figure}[h]
\centering
\includegraphics[width=.3\textwidth]{RGB_Canales}
\caption{Los tres canales del espacio RGB \cite{Manual:HAE}}
\label{fig:3.2}
\end{figure}
La combinación de estos colores crea todas la gama de colores representable.
El valor de la intensidad de cada canal depende de la codificación usada para su representación (Ocho bits dan Dieciséis millones de colores) como se muestra en la figura  \ref{fig:3.3}

\begin{figure}[h]
\centering
\includegraphics[width=.3\textwidth]{RGB}
\caption{Representación del modelo RGB\cite{Manual:HAE}}
\label{fig:3.3}
\end{figure}

\subsection{HSV}
El modelo HSV \cite{modelo:hsv} esta orientado a la descripción de los colores en términos mas prácticos para el ser humano que el RGB, los canales significan algo mas que la cantidad de cada color, por lo que es mas practico para el ser humano.Lo que se observa en la figura \ref{fig:3.4}.
 
\begin{itemize}
	\item H: (Matiz) que representa el tono o color.\\
	\item S: (Saturación) representa el nivel de saturación de un color.\\
	\item V: (Brillo) representa la intensidad lumínica.\\
\end{itemize}

\begin{figure}[h]
\centering
\includegraphics[width=.35\textwidth]{HSV}
\caption{Representación del modelo HSV\cite{Manual:HAE}}
\label{fig:3.4}
\end{figure}

Una ventaja con otros espacios de color parecidos es que este permite representar todas las combinaciones del espacio RGB.

\section{Transformada de Hough }

Uno de los puntos relevantes del proyecto es la detección de las líneas pintadas o detectadas por el algoritmo (modo automático) para ello vamos a usar una técnica que sirve para detectar formas, expresadas de forma matemática, dentro de imágenes.

Esta técnica fue inventada por Richard Duda y Peter Hart en 1972 pero diez años antes, Paul Hough propuso y patentó \cite{pat:patHough} la idea inicial de detectar líneas en la imagen. Mas tarde se generalizó para detectar cualquier figura.

\subsection{Teoría}

Normalmente para detectar figuras sencillas en una imagen primero hay que usar algún algoritmo de detección de bordes o una binarización de la imagen, quedándonos con la región de interés apropiada (los píxeles que forman las rectas) pero normalmente faltan píxeles por el ruido en la imagen.

Para ello el método de Hough propone solucionar el problema detectando grupos de puntos que forman los bordes de la misma figura y así conseguir unirlos creando la recta real a la que pertenecen.

\subsection{Psudocódigo Transformada de Hough}
Como podemos ver en el siguiente Psudocódigo 
\ref{alg:Hough}.

%Hough
\begin{algorithm*}
\DontPrintSemicolon
\KwIn{Imagen}
\KwOut{(list) de segmentos encontrados }

%$ S = \varnothing $

\ForEach {punto en la imagen}{
		\If {punto \textbf{(x,y)} esta en un borde:}{
			\ForEach {angulo en ángulos $\Theta$ }{
				Calcular $\rho$  para el punto (x,y) con angulo $\Theta$ \;
				Incrementar la posición ($\rho$ , $\Theta$ ) en el acumulador\;
			}
		}
}
Buscar las posiciones con mayores valores en el acumulador\;	

\Return {$rectas$ Las rectas cuyos valores son los mayores en el acumulador}
\label{alg:Hough}
\end{algorithm*}

\subsection{Limitaciones}
Para que este proceso sea exitoso, los bordes del objeto deben ser detectados correctamente con un buen pre-procesado de la imagen y aparecer claramente las nubes de puntos que forman las rectas.
Como se muestra en el figura \ref{fig:3.5}.

\begin{figure}[h]
\centering
\includegraphics[width=.3\textwidth]{Hough}
\caption{Líneas de Hough\cite{opencv:HoughIm}}
\label{fig:3.5}
\end{figure}
\subsection{Transformada probabilística de Hough}

Tal y como se explica en \cite{Kiryati20001157}, la transformada probabilística de Hough es una version que se basa en que la detección de bordes o la producción de la imagen binaria que contiene el objeto, podría tener ruido y por lo tanto los píxeles que corresponden al ruido con la transformada normal podrían ser considerados como una recta, cuando en verdad es ruido.\\

Para que unos puntos sean considerados recta en la transformada probabilística, es necesario menos puntos que en la transformada normal de Hough.
Pero este método penaliza a los puntos que se encuentran aleatoriamente por toda la imagen (ruido) frente a los que se localizan perfectamente agrupados formando las rectas. 
Un exceso de ruido en la imagen también haría este método inservible, pero para pequeñas cantidades lo hacen mas preciso que el método normal.
Otra ventaja es que con este método obtenemos el segmento que necesitamos, no la prolongación de él hasta el infinito.

\section{Skeletonize }
Tal y como se explica en \cite{scik:skeleton}, dentro del pre-procesado de la imagen, uno de los puntos clave para que nuestro método funcione.
Después de su binarización y la detección de los bordes de la imagen a procesar, debemos reducir la región sobre la que aplicar la transformada de Hough, así conseguiremos que esta sea mas rápida, y detecte menos número de líneas imaginarias por cada línea real.

Esto lo conseguiremos usando una función de esqueletonizado que nos devuelve lo que su nombre indica el esqueleto de los bordes de la imagen reducidos a 1 pixel, Como podemos observar en la imagen \ref{fig:3.6}.


\begin{figure}
\begin{subfigure}[b]{.5\linewidth}
\centering\large \includegraphics[width=.9\textwidth]{skeletonizeA}
\caption{Original}
\end{subfigure}%
\begin{subfigure}[b]{.5\linewidth}
\centering\large \includegraphics[width=.9\textwidth]{skeletonizeB}
\caption{Skeletonizada}
\end{subfigure}
\caption{Ejemplo de eskeletonize.}\label{fig:3.6}
\end{figure}


\section{Grafos}	
\subsection{Introducción}
La teoría de grafos\cite{Wiki:Grafos} en un campo dentro de la computación y de las matemáticas, estudia las propiedades de los grafos que están compuestos por vértices y aristas, que comunican estos los vértices.
Es una rama muy amplia pero solo vamos a centrarnos en uno de los problemas que podemos resolver gracias a estas teorías.

\subsection{K-componentes}
Una de las propiedades del grafo con sus componentes que nos indica cómo de fuertemente conexos están sus vértices, gracias a esta teoría nos aprovechamos que cuando un grafo esta dividido en clusters, agrupaciones fuertemente conexas de parte de sus vértices, por el problema de los k-componentes podemos saber que conjuntos de vértices forman los clusters por lo tanto, aplicado a nuestro problema los vértices que tengan aristas que los comuniquen pertenecerán a las mismas rectas y de cada conjunto de vértices sacaremos una recta.

\section{Bordes}
En lo que nos estamos basando, para poder resolver el problema,
es en la detección de bordes mediante kernels conocidos.
Para poder detectar donde esta el cambio de color en el histograma y esos cambios bruscos de pigmentación son los que nos indican que eso es un borde.
Para ello se han desarrollado numerosos métodos matemáticos que pasando una mascara, por toda la imagen, nos devolverían otra imagen con los bordes llamada mascara.

Como en los anexos donde usamos toda la variedad de ellos que hemos encontrado aquí simplemente los nombraremos.

Kernel \cite{wiki:kernels}: Un kernel es una matriz que se operara con una porcion de pixeles para suavizar o detectar algún elemento que sea de nuestro interés.

\begin{itemize}
\item Laplace \cite{wiki:Laplace}.
\item Prewitt \cite{wiki:Prewitt}.
\item Scharr \cite{jon:Scharr}.
\item Sobel \cite{wiki:Sobel}.
\item Roberts \cite{wiki:Roberts}.
\item Kirsch \cite{wiki:Kirsch}.
\item Gabor \cite{wiki:Gabor}.
\end{itemize}
Aparte de los métodos antes mencionados hemos combinado distintos formas para detectar bordes através de los autovectores \cite{wiki:Eigenvector} largos, de la matriz Hessiana \cite{wiki:Hessiana}.
\capitulo{4}{Técnicas y herramientas}

%\maketitle
\section{Interfaz gráfica de usuario:}
\subsection{Tkinter}
Es una librería que proporciona el diseño y visualización de interfaces de usuario en Python. Esta a su vez esta basada en librerías de TK/TCL que están incluidas por en la propia instalación.
\\

\textbf{Ventajas:}

\begin{itemize}
\item Es fácil de usar y es recomendable para el aprendizaje del lenguaje.

\item Viene preinstalado con la distribución por lo que su uso es inmediato

\item Al servir para aprendices y venir preinstalado podemos encontrar multitud de tutoriales y de documentación sobre ello.
\end{itemize}

\textbf{Desventajas:}

\begin{itemize}
\item Pocos elementos gráficos. Escaso control de las ventanas y bastante lento.
\end{itemize}

\subsection{WxPython}
Es una librería basada en otra importante que veremos mas adelante, también es multiplataforma y esta programada en C/C++, es mas nueva que Tkinter.
\\

\textbf{Ventajas:}

\begin{itemize}
\item Es más difícil de usar que Tkinter pero aun así hay mucha documentación sobre ella.

\item Dispone de gran cantidad de elementos gráficos por lo que es bastante potente aunque con alguna limitación respecto a otras.

\item Permite hacer una barrera o separación entre el código Python y lo que es la interfaz.
\item Cuenta con una gran comunidad de gente que lo usa y publica ejemplos y tutoriales.
\end{itemize}

\textbf{Desventajas:}

\begin{itemize}
\item La principal desventaja es que se actualizan las versiones mucho y para mantener una aplicación durante largo tiempo perdemos tiempo de.
\item Es más complejo de usar que el anterior.
\end{itemize}

\subsection{PyQt}
Es más difícil de usar que WxPython y WxWidget pero da más control sobre los elementos gráficos y muchas librerías se basan en ello, por lo que se puede encontrar bastante información sobre esta librería.
\\

\textbf{Ventajas:}

\begin{itemize}
\item Al ser tan usada, si instalamos Python desde Anaconda, que es un cojunto de librerías y aplicaciones de Python, ya tendríamos PyQt4 por defecto instalado.

\item Podemos usar un IDE con el que estamos muy familiarizados en la carrera y que funciona muy bien (Eclipse) pero tenemos que instalar un plugin para Python (PyDev), ya que es un IDE basado en java también deberíamos instalar Java 8.
\end{itemize}

\textbf{Desventajas:}

\begin{itemize}
\item Es mas difícil de entender y comprender que los anteriores pero quedan mas limpias las interfaces.
\item Si somos puristas e instalamos Python solo sin IDE ni nada no vendría instalado pero si usamos Anaconda si que vendría instalado
\end{itemize}

\subsection{WxWidgets:}
Como hemos mencionado anteriormente es muy parecido a WxPython por lo que no vamos a entrar en detalle también está programado en C/C++ y es multiplataforma da un aspecto de comportamiento nativo.
\\


\section{Plantillas}
Las plantillas sirven para generar código con la sustitución de los valores dentro de sus variables y así tener siempre un código con los valores en la ejecución.\\
Nosotros las vamos a usar en combinación con LaTex para generar el pdf con el informe de las estadísticas de la ejecución del algoritmo que detecta las líneas.\\
Como podemos observar hay muchos frameworks para usar las plantillas pero hemos elegido comparar varios usados otros años por compañeros.
\subsection{Mustache}
Pystache es una implementación de Mustache en Python para el diseño de plantillas, está inspirado en Ctemplate y et.
En estos tipos de lenguaje no se puede aplicar lógica de aplicación simplemente son para una presentación más fluida.

\textbf{Ventajas:}

\begin{itemize}
\item Tiene documentación online.
\item Está disponible para una gran variedad de lenguajes de programación.
\item Si necesitásemos hacer documentos xml seria la herramienta perfecta.
\end{itemize}

\textbf{Desventajas:}
\begin{itemize}
\item Carece de ejemplos o tutoriales de cómo usarlo con LaTex en su versión de Python.
\end{itemize}


\subsection{Jinja2}

El nombre de la librería viene del templo Japonés Jinja.\\
Es una librería para escribir plantillas, para Python, funciona con las últimas versiones de Python, también es una de las más usadas librerías está inspirada en Django.


\textbf{Ventajas:}
\begin{itemize}
\item Viene preinstalado con Anaconda ya que es uno de los más utilizados.
\item Fácil de usar y de pasar los parámetros.
\item Fue creada para Python2 pero también funciona en Python3
\end{itemize}
\textbf{Desventajas:}
\begin{itemize}
\item No esta implementado más que para Python.
\end{itemize}

\subsection{Comparación de las herramientas}
Esta comparación son dos columnas\ref{tb:1} sacada de la tabla comparativa de herramientas obtenida en wikipedia.


\begin{table}[]
\centering
\caption{Tabla comparativa de herramientas}
\label{tb:1}
\resizebox{\textwidth}{!}{
\begin{tabular}{|l|l|l|l|l|l|l|l|l|l|l|l|l|}
\hline
Biblioteca & Lenguaje & Licencia & Variables & Funciones & Include & \begin{tabular}[c]{@{}l@{}}Inclusiones\\ condicionales\end{tabular} & Bucles & Evaluacion & Asignaciones & Errores y excepciones & \begin{tabular}[c]{@{}l@{}}Plantillas\\ naturales\end{tabular} & Herencia \\ \hline
Jinja2     & Python   & BSD      & SI        & SI        & SI      & SI                                                                  & SI     & SI         & SI           & SI                    & SI                                                             & SI       \\ \hline
Mustache   & +30      & MIT      & SI        & SI        & SI      & SI                                                                  & SI     & NO         & NO           & SI                    & SI                                                             & NO       \\ \hline
\end{tabular}
}
\end{table}
\section{IDE:}
Un IDE es una aplicación para poder desarrollar código y ejecutar pero facilita la navegabilidad de por los paquetes y poder debugear de forma mas eficiente que sin el.\\
Es una herramienta indispensable para el desarrollo de software,un IDE se compone normalmente de un editor de código fuente, un depurador, la mayoría de ellos tienen también autocompletado,compilador, indentador de código e interprete.\\
Ventajas:
\begin{itemize}
	\item Maximizar la productividad.
	\item Juntar todo el ciclo de desarrollo en una herramienta.
	\item Algunos soportan múltiples lenguajes. 
	\item Sin ello es difícil leer código y editarlo.
\end{itemize}

\subsection{Eclipse:}
Es una aplicación software con muchos plugins y herramientas para el desarrollo de software, esta basado en Java, para su ejecución e instalación es necesario tener Java instalado.\\
Fue desarrollado por IMB en un principio pero ahora esta desarrollado por la Fundación Eclipse, una fundación independiente y sin animo de lucro.


Podemos acceder a la pagina atraves del siguiente enlace: \url{https://eclipse.org/home/index.php}

\textbf{Ventajas:}
\begin{itemize}
	\item Lo hemos usado en múltiples asignaturas durante la carrera.
	\item Es gratuito por lo que podemos descargarlo sin problemas.
	\item Es un programa muy completo y que es sencillo.
\end{itemize}

\subsection{PyDev:}
Es un plugin para poder usar el IDE de Eclipse para programar en Python, Jython e IronPython.
Se instala desde eclipse desde: Help > Install New Software.

Podemos encontrarlo a atraves del siguiente enlace:\cite{Eclipse:PyDev}

\section{Modelado}
El modelado es la creación de diagramas que nos explican que apariencia y comunicación van a tener los datos entre ellos, por eso es necesaria una herramienta especializada en ello, hay muchas en el mercado pero en las asignaturas de la carrera dimos una con la que ya estamos familiarizados, por eso no vamos a analizar todas las disponible y comparar, simplemente vamos a usar la que ya conocemos como funciona y que para las siguientes parte y diagramas ya nos va a servir.

\subsection{Astah}
Es una herramienta de modelado UML para creación de diagramas orientados a objetos.
Podríamos hacerlos dibujando pero es mas preciso y mas profesional usar una herramienta que esta pensado para ello.
Podemos encontrar la herramienta através del siguiente enlace: \cite{Modelado:Astah}

\subsection{XML:}
Para guardar los nombres de los ficheros que se generan al guardar un proyecto, vamos a usar un archivo XML, que contenga los nombres de los ficheros que se generan.\\
XMl significa Extensible Marking Language, también otra de sus propiedades es que es un lenguaje sencillo que estructura el contenido por medio de :
\begin{itemize}
\item Cuerpo: Es obligatorio en su sintaxis, normalmente contiene un elemento raíz del que cuelgan todos los demás.
\item Elementos: Pueden tener mas elementos, cadenas de caracteres o nada.
\item Atributos: Los contienen los elementos y son sus características o propiedades.
\item Secciones CDATA: Para especificar caracteres especiales sin que rompan la estructura XML.
\end{itemize}
 
Para mas informacion respecto a ello podemos encontrarlo aqui:\cite{Wiki:xml}.

En nuestro caso lo vamos a usar como un fichero donde se guardan todo lo que contendría un proyecto.

\capitulo{5}{Aspectos relevantes del desarrollo del proyecto}

\section{Entorno de desarrollo}
Como entorno de desarrollo de los prototipos hemos designado Jupyter ya que en sus notebooks interactivos puedes ejecutar directamente código python como si fuese un interprete.\\

\subsection{Ventajas}
\begin{itemize}
\item He podido añadir widguets para calibrar en buen grado las funciones que hemos utilizado.\\
Gracias a estos widguets podemos dar valores e ir viendo como cambia la salida de la función de forma interactiva.

\begin{figure}[h]
\centering
\includegraphics[scale=0.75]{Widget}
\caption{Ejemplo de un widguet sobre la función de hough}
\end{figure}

\item Su rápida visualización sin tener grandes conocimientos de interfaz gráfica ha sido un gran apoyo para poder visualizar desde el principio las imágenes procesadas y como quedaban.\\ 

\begin{figure}[h]
\centering
\includegraphics[scale=0.55]{ComparativaLineas}
\caption{Ejemplo de una visualización del resultado intermedio de las funciones.}
\end{figure}

\item Desde el propio entorno puedes ejecutar no solo código estructurado en script sino también código estructurado en clases y llamadas a métodos es como un IDE pero con limitaciones.

\begin{figure}[h]
\centering
\includegraphics[scale=0.75]{Ejecucion}
\caption{Ejemplo de una Ejecución.}
\end{figure}

\item Multitud de librerías y funciones que en entornos parecidos como matlab serian de pago y aquí al ser software libre el ejemplo anterior lo resume en una librería numpy \cite{Numpy}.
\end{itemize}

\section{Procesado imagen}
Para llegar a conseguir calcular las lineas que había pintadas en las imágenes tube que realizar una serie de pasos que vamos a resumir en tres etapas.
\subsection{Binarización}
Partiendo de una imagen que solo tenia lineas en rojo pintadas encima de las estrías producidas por el desgaste y lo demás de la imagen en escala de grises, lo primero fue leer la imagen a trabes de las funciones ya programadas en la librería de Scikit-Image(skimage).\\
Una vez que tenemos la imagen guardada en el espacio de color RGB podemos empezar el procesado quedándonos con el canal Rojo.\\
Calculamos la distancia de cada pixel de la imagen al color rojo restando, uno menos el valor absoluto del pixel en el canal S (saturación) del espacio de color HSV , (restando el valor absoluto a la unidad conseguimos normalizar entre [0-1])y pasamos la imagen de distancias a blanco y negro y así tendremos un valor entre 0 y 256 en cada pixel correspondiente a la distancia al rojo cuanto mas alejado mas negro y los que sean rojos en blanco.\\
Para que la diferencia sea blanco o negro binarizamos la imagen con un valor umbral calculado como threshold otsu y así la imagen los valores de la distancia que sean mayores que el umbral pasaran a valer máximo y los que no consigan pasar el umbral serán los bordes(blanco).

\subsection{Obtener segmentos}
Partimos de la imagen binarizada y lo primero es reducir el grosor de las lineas detectadas a 1 pixel eso lo conseguimos llamando a la función skeletonize que nos devuelve la imagen con las lineas de 1 pixel (así no acumulamos errores y es mas rápida la búsqueda de rectas).\\
Seguidamente llamamos a la función "probabilistic hough line" que nos va a encontrar segmentos que formaran las lineas el funcionamiento ha sido explicado en el apartado (conceptos teóricos).

\subsection{Procesado de segmentos}
Llegados a este punto lo que tenemos son muchos segmentos que forman las lineas reales y tenemos que unirlos.\\
Para ello vamos a usar la teoría de grafos añadiendo los segmentos a un grafo.\\
Para unir dos segmentos tiene que cumplirse que la distancia mínima entre sus extremos sea menor que un umbral y si pasa este punto comprobaremos que el angulo que forman entre ellas sea menor a otro umbral y si cumplen las dos condiciones añadiremos un camino al grafo desde la recta 1 a la recta 2.
\subsection{Recuperación de lineas}
Ahora lo que tenemos es un grafo con clusters ya que cada cluster se identifica con únicamente una recta y tendremos tantos como rectas.
Un problema de grafos es el problema de las k-componentes pero a nosotros solo nos interesan las 1-componentes del grafo ya que cada grupo de estos segmentos "cercanos" se corresponde con una recta real.
devolvemos la combinación de los segmentos mas relevantes de cada cluster y estos se convierten en nuestra buscada linea real.

\subsection{Resumen pasos}

\begin{itemize}
\item Binarizar la imagen para solo quedarnos con los objetos de interés
\item Obtener segmentos que forman trozos de las lineas.
\item Añadir caminos entre las lineas cercanas en un grafo
\item obtener los grupos de lineas próximas y devolver la recta que las una. 
\end{itemize}


%Este apartado pretende recoger los aspectos más interesantes del desarrollo del %proyecto, comentados por los autores del mismo.
%Debe incluir desde la exposición del ciclo de vida utilizado, hasta los %detalles de mayor relevancia de las fases de análisis, diseño e implementación.
%Se busca que no sea una mera operación de copiar y pegar diagramas y extractos %del código fuente, sino que realmente se justifiquen los caminos de solución %que se han tomado, especialmente aquellos que no sean triviales.
%Puede ser el lugar más adecuado para documentar los aspectos más interesantes %del diseño y de la implementación, con un mayor hincapié en aspectos tales como %el tipo de arquitectura elegido, los índices de las tablas de la base de datos, %normalización y desnormalización, distribución en ficheros3, reglas de negocio %dentro de las bases de datos (EDVHV GH GDWRV DFWLYDV), aspectos de desarrollo %relacionados con el WWW...
%Este apartado, debe convertirse en el resumen de la experiencia práctica del %proyecto, y por sí mismo justifica que la memoria se convierta en un documento %útil, fuente de referencia para los autores, los tutores y futuros alumnos.

\capitulo{6}{Trabajos relacionados}

En esta sección, vamos a hablar de otros trabajos que están relacionados, tanto en la resolución del problema nuestro en particular, como resolver el problema de las detecciones de otro tipo de estrías o información.

\section{Artículo de Alejandro Pérez Pérez}
El artículo \cite{Lalueza:perez}, es en el que se basan todos los artículos que hay hasta el momento, para resolver el problema de deducir la dieta a partir de las marcas dentales, lo mencionamos por su importancia.

Tras un análisis de diversas poblaciones con alimentaciones distintas, ha agrupado en diversos conjuntos cada una de ellas. De este modo se pude saber si un individuo fue, recolector y cazador de climas tropicales, recolector y cazador de climas aridos, agricultor o recolector y cazador carnívoro.
Esto lo estimas mediante unas funciones propuestas por ellos.

\section{Artículo de Rebeca García González}
El articulo \cite{Rebeca:garcia} de nuestro cliente nos muestra como basandose en el articulo \cite{Lalueza:perez} de Pérez Pérez, hace lo mismo para clasificar un individuo obtenido de la cueva <<El Mirón>>, esta situada en la comunidad autónoma de Cantabria, España.

A la muestra anteriormente mencionada se le aplicara el análisis de las estrías de dieta, sobre una imagen del premolar de abajo a la izquierda, P4 de la dentadura. 
Una vez que hacen este artículo, encuentran interesante el proponer una herramienta que a partir de las imágenes que ya tienen pintadas, pudiera automatizar el trabajo repetitivo, obteniendo las estrías y calculando las estadísticas para nuevos individuos.

\section{Trabajo fin de grado: Sergio Chico Carrancio}
Podemos acceder al proyecto de Sergio en : \url{https://github.com/Serux/perikymata}.
En este proyecto se centraron en detectar estrías en las piezas dentales, pero no eran las mismas que nosotros queremos detectar, sino unas estrías llamadas perykimata, estas muestran el crecimiento del diente. Pero muchas de las funcionalidades y partes del programa, son parecidas a las nuestras, aunque solucionado de formas distintas.
Sergio uso OpenCv, es una librería de funciones para tratamiento de imágenes, para java para detectar las perykimata y su aplicación seguía esta serie de pasos:
\begin{itemize}
\item Unir las imágenes: El no tenia la imagen completa sino porciones de imágenes que debía unir antes para poder empezar.
\item Aplicar filtros: Sobre la imagen ya unida aplica filtros para resaltar sobre el fondo negro las perikymata.
\item Para finalizar: El usuario debe marcar una línea sobre la imagen, tratando de que pase por encima del mayor numero de perikymata, una vez hecho esto las perikymata detectadas, se marcan automáticamente permitiendo modificar o corregir errores en el proceso.
\end{itemize}

\capitulo{7}{Conclusiones y Líneas de trabajo futuras}


En este apartado, vamos a mencionar algunas conclusiones a las que hemos llegado, después de realizar el proyecto, y algunas partes sobre las que deberían recalcar mas adelante, direcciones a tomar para futuros desarrolladores que continuasen el proyecto.



\section{Conclusiones}
Para desarrollar este apartado vamos a ir separando por puntos las conclusiones, dependiendo de la temática y de los campos a los que se refieran.
Pero van a ir mencionando todas las conclusiones a las que hemos llegado, cubriendo todos los puntos mas relevantes que hemos realizado.

\begin{itemize}
\item Dinámica del proyecto: Para concluir con el proyecto, estoy muy satisfecho con la dinámica del proyecto, y con todos los aspectos que hemos tocado dentro del mismo.
Algunos aspectos de investigación han sido difíciles, no hay demasiada información respecto al tema, y somos los primeros en hacer una aplicación para resolver estos problemas, por ello nos hemos encontrado con impedimentos que hemos solucionado invirtiendo mas tiempo.

\item Transparencia al usuario: El reto mas grande ha sido conseguir realizar todas las funciones con la mas sencilla transparencia para el usuario. Ante todo hemos invertido parte del tiempo, no solo en resolver sino en pensar cual sería la forma mas sencilla, para que un usuario use el programa intuitivamente. 

\item Procesado y extracción de características: Los algoritmos para el procesado y la reconstrucción de las características que necesitábamos obtener, no eran directos. Como la variabilidad de tantos parámetros afectaba enormemente al resultado final, hemos tenido que ir testando muchos valores, para al final generalizarlos y conseguir un resultado bueno.

\item Uso del conocimiento adquirido durante la carrera: Durante el proceso de desarrollo hemos echo uso de muchos conceptos disjuntos que hemos dado durante la carrera y juntándolos para conseguir nuestras necesidades.

\item Lenguaje de desarrollo: El lenguaje de desarrollo, Python, es un plus para el proyecto ya que al investigar sobre este, hemos visto que últimamente en investigación y procedimientos matemáticos esta al alza, dispone de una gran cantidad de librerías con funciones muy actuales y de fácil aplicación. La programación es mas eficiente que en otros lenguajes.

\item Agilidad: Para completar este punto yo he tenido una curva de desarrollo muy estable y muy alta, ya que me he centrado exclusivamente a este proyecto.
Por otra parte como agilidad en lo referido a metodología de desarrollo Scrum \cite{Scrum}, ha sido muy útil, porque en cada Sprint teníamos, una version con algunas capacidades añadidas, pero usable desde el primer momento.

\item Valor adquirido para mí: 
El principal motivo para la elección de este proyecto fue, el tema del mismo, es algo que si no se hace en colaboración con una universidad no es muy probable que surja. Otro de los motivos fue el propio campo de la antropología, es algo que siempre me ha gustado y nunca tuve la oportunidad de conocer mas a fondo.
También como lenguajes de desarrollo elegí Python, porque en este lenguaje no sabía como hacer interfaces gráficas y como lo hemos dado en varias asignaturas quería terminarlo de completar sabiendo usar las interfaces ya que en contraposición en Java si que he visto y tocado en varias ocasiones el desarrollo de interfaces.

\item Los patrones de diseño:
Después de hacer este proyecto y las dificultades que han surgido, me he dado cuenta que las asignaturas de diseño debería de haberlas cursado, porque me han resuelto muchos problemas y me han ahorrado mucho tiempo.
Pero no viene de mal mencionar que al final las he usado y las he aprendido porque he querido completar mi formación.

\end{itemize}



\section{Líneas de trabajo futuras}
Como partes de mejora futuras podríamos enumerar varios aspectos del proyecto.

\begin{itemize}
	\item Detección automática de bordes: 
Podríamos proponer a incluir, un algoritmo mas especializado en la detección de bordes, ya que las limitaciones que nos hemos encontrado hacen que este sea el puto mas flojo del proyecto, como propuesta, podría ser la creación desde cero de un algoritmo que detectase los bordes de una manera mas precisa.

	\item Unión de segmentos: 
La transformada de Hough para la detección de las lineas ha dado muy buenos resultados, pero creo que se podría mejorar con este algoritmo: 

	\begin{itemize}
		\item Binarizar como hasta ahora.
		\item Esqueletonizar la imagen como hasta ahora
		\item Detectar las lineas mas grandes en la imagen y borrarlas de la mascara.
		\item Detectas las lineas bastante mas pequeñas pero aun no la mas cortas del todo y borrarlas de la máscara.
		\item Detectar las lineas cortas en la imagen y borrarlas de la máscara
		\item Filtrar las lineas que en la mascara no contengan mas de un 70\% de blanco. (Así quitamos falsos positivos).
	\end{itemize}
	\item Añadir mas tipos de estrías que podríamos detectar como modos auxiliares para determinar que características queremos extraer de la imagen. Es decir acoplar en la aplicación la resolución de un nuevo problema de detección.
	
\end{itemize}


\bibliographystyle{plain}
\bibliography{bibliografia}

\end{document}