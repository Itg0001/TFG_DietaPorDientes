\capitulo{1}{Introducción}
%\maketitle

Para ponernos en contexto sobre que va a hacer y en que va a consistir el proyecto, vamos a explicar brevemente que hará.

El problema planteado, consiste en crear una herramienta o aplicación para detectar las estrías que se producen en los dientes, al masticar distintos tipos de comida.

Para ser mas exactos, todos los materiales tienen una dureza tangible, los dientes tienen una dureza superior a la de los alimentos, sin embargo  algunas partes de los alimentos tienen mas dureza que los dientes, por lo que los consiguen rayar, creando unas estrías que dependiendo de los alimentos siguen distintos patrones.

 
Vamos a detectar esas estrías, pero en el contexto de la prehistoria, para después poder clasificar los individuos en función de la dieta que llevaban.

Nuestro proyecto va a tener una parte de visión artificial, que detecte las lineas que han pintado, sobre las imágenes de microscopio de electrones a 100 aumentos, otra parte de diseño e implementación, del software a usar y distintos análisis de las herramientas que usaremos para ello.

La herramienta a usar tendrá implementados 3 modos claramente diferenciados.

\begin{itemize}
\item Detección automática de las lineas pintadas sobre la imagen.
\item Pintar lineas manualmente sobre la imagen o corregir las posibles lineas que el anterior modo no consiga detectar.
\item Automático que buscara las estrías por la imagen.Este modo quedara como extra, ya que hemos tenido una gran agilidad realizando el proyecto, vamos a tratar esta parte para completar e investigar sobre detección automática de bordes.
\end{itemize}








 