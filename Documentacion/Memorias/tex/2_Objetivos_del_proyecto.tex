\capitulo{2}{Objetivos del proyecto}

Este apartado explica de forma precisa y concisa cuales son los objetivos que se persiguen con la realización del proyecto. Se puede distinguir entre los objetivos marcados por los requisitos del software a construir y los objetivos de carácter técnico que plantea a la hora de llevar a la práctica el proyecto.

\section{Objetivos:}
Esta parte se corresponde con el esquema de todos los puntos.
\begin{itemize}
\item Analizar el problema planteado por Rebeca y buscar una solución:
	\begin{itemize}
		\item Primero tenemos que documentarnos y saber que son las lineas a detectar.
		\item Buscar una solución para detectar dichas estrías pintadas, de forma automática, ya que hasta ahora era un problema manual que tenían que medir a mano.
		\item Permitir que se pinten através de la aplicación para futuras imágenes.
		\item Iniciar la forma de detección completamente automática.
	\end{itemize}
\item Crear un notebook para el procesado:
	\begin{itemize}
		\item Crear un prototipo interactivo atrevés de los notebooks de jupyter.
		\item Procesar la imagen para obtener la mascara o imagen binarizada con las líneas.
		\item Detectar los segmentos que se corresponden con las líneas.
		\item Juntar los segmentos para obtener líneas reales.
	\end{itemize}
\item Crear la aplicación:
	\begin{itemize}
		\item Crear una interfaz gráfica con la imagen,y botonería.
		\item Añadir a la gui el modo 1: Buscar estrías pintadas.
		\item Añadir a la gui el modo 2: Permitir pintar estrías.
		\item Añadir a la gui el modo 3 o a un notebook: Completamente automático.
	\end{itemize}
\end{itemize}

\section{Resumen}

\subsection{Analizar y resolver el problema:}
El problema va a consistir en detectar las lineas que tienen ya pintadas, en las imagenes ,para poder automatizar dicho proceso, ya que los pasos anteriores eran:
\begin{itemize}
\item Pintar las estrías encima de la imagen.
\item Obtener de las estrías ,de forma manual, su tamaño, angulo,dirección.
\end{itemize}
Y los pasos através de nuestra aplicación son:
\begin{itemize}
\item Abrir la imagen y seleccionar el color de las lineas
\item Dar al botón de calcular las lineas y guardarlo: Tendíamos el CSV con los estadísticos y atributos de ellas.
\end{itemize}
Como podemos observar nuestros pasos nos devuelven las estrías de forma mucho mas cómoda y mas rápida que buscándolas a mano.

\subsection{Crear un prototipo} 
En esta parte hemos pensado que seria mas cómodo, antes de ponernos a diseñar o implementar, comprobar que lo que tenemos pensado para resolver el problema que funcione.

Crear un notebook de jupyter no nos exige que programemos nada de la GUI, al ser interactivo, fuimos haciendo todos los pasos necesarios para resolver el problema y una vez conseguido. Como producto tenemos el núcleo de calculo y pasamos a hacer el diseño.

\subsection{Crear la aplicación}
Llegados a este punto ya tenemos todo lo necesario para hacer nuestras clases y nuestra aplicación.

Así que ya tenemos todo lo necesario y solo queda el producto entregable que sera la aplicación enlazada con la GUI junto con, la creación de botones, pestañas , ventanas con sus respectivas implementaciones y funcionalidades.