\capitulo{2}{Objetivos del proyecto}

Este apartado explica de forma precisa y concisa cuales son los objetivos que se persiguen con la realización del proyecto. Se puede distinguir entre los objetivos marcados por los requisitos del software a construir y los objetivos de carácter técnico que se plantean a la hora de llevar a la práctica el proyecto.

\section{Objetivos}
A continuación se mostrara el esquema con todos los puntos a tratar en este proyecto.
\begin{itemize}
\item Analizar el problema planteado por Rebeca, nuestra colaboradora del Laboratorio de Evolución Humana, y buscar una solución:
	\begin{itemize}
		\item Primero tenemos que documentarnos, y saber que son las líneas a detectar.
		\item Buscar una solución para detectar dichas estrías pintadas, de forma automática, ya que hasta ahora era un problema manual que tenían que medir a mano.
		\item Permitir que se pinten a través de la aplicación para futuras imágenes.
		\item Iniciar la forma de detección completamente automática.
	\end{itemize}
\item Crear un notebook para el procesado:
	\begin{itemize}
		\item Crear un prototipo interactivo a través de los notebooks de jupyter.
		\item Procesar la imagen para obtener la máscara o imagen binarizada con las líneas.
		\item Detectar los segmentos que se corresponden con las líneas.
		\item Juntar los segmentos para obtener líneas reales.
	\end{itemize}
\item Crear la aplicación:
	\begin{description}
		\item Crear una interfaz gráfica con la imagen y botonería.
		\item [Modo 1] Modo semi-automático, detección, conteo y análisis de estrías.
		\item [Modo 2] Permitir pintar estrías y corrección de segmentos.
		\item [Modo 3] Completamente automático.
	\end{description}
\end{itemize}

\section{Resumen}

El problema va a consistir en detectar las líneas que tienen ya pintadas y detectar desde cero las no pintadas, en las imágenes, para poder automatizar dicho proceso, ya que los pasos anteriores eran:
\begin{itemize}
\item Pintar las estrías encima de la imagen.
\item Obtener de las estrías, de forma manual, su tamaño, ángulo, dirección.
\end{itemize}
Y los pasos a través de nuestra aplicación son:
\begin{itemize}
\item Abrir la imagen y seleccionar el color de las líneas
\item Dar al botón de calcular las líneas y guardarlo: Tendíamos el CSV con los estadísticos y atributos de ellas.
\end{itemize}
Como puede intuirse, nuestros pasos nos devuelven las estrías de forma mucho mas cómoda y mas rápida que buscándolas a mano.

\subsection{Crear un prototipo} 
En esta parte hemos pensado que sería más cómodo, antes de ponernos a diseñar o implementar, comprobar que lo que tenemos pensado para resolver el problema funcione.

Crear un Notebook de Jupyter (ver sección \ref{notebook:jupiter}), no nos exige programar nada de la GUI, al ser interactivo, vamos a implementar todos los pasos necesarios para resolver el problema. 
Una vez conseguido, como producto tenemos el núcleo de cálculo y pasaremos a hacer el diseño.

\subsection{Crear la aplicación}
Llegados a este punto ya tendremos todo lo necesario para hacer nuestras clases y nuestra aplicación.

Así que ya tenemos todo lo necesario y solo queda el producto entregable que será la aplicación enlazada con la GUI junto con la creación de botones, pestañas, ventanas con sus respectivas implementaciones y funcionalidades.