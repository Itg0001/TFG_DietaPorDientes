\capitulo{5}{Aspectos relevantes del desarrollo del proyecto}

\section{Entorno de desarrollo}
Como entorno de desarrollo de los prototipos hemos utilizado Jupyter, para los prototipos, ya que en sus notebooks interactivos puedes ejecutar directamente código Python como si fuese un interprete.
Y Eclipse mas PyDev, como IDE, para programar en clases y paquetes de forma mas cómoda.
\subsection{Ventajas}
\begin{itemize}
\item He podido añadir widgets para calibrar en buen grado las funciones que hemos utilizado.\\
Gracias a estos widgets podemos dar valores e ir viendo como cambia la salida de la función de forma interactiva. Como podemos observar en la imagen \ref{fig:5.1}

\begin{figure}[h]
\centering
\includegraphics[width=0.65\textwidth]{Widget}
\caption{Ejemplo de un widget sobre la función de Hough}
\label{fig:5.1}
\end{figure}

\item Su rápida visualización sin tener grandes conocimientos de interfaz gráfica ha sido un gran apoyo para poder visualizar desde el principio las imágenes procesadas y como quedaban. Como podemos observar en la imagen \ref{fig:5.2} 

\begin{figure}
\begin{subfigure}[b]{.5\linewidth}
\centering\large \includegraphics[width=.9\textwidth]{ComparativaLineas2}
\caption{Líneas antes de unir}
\end{subfigure}%
\begin{subfigure}[b]{.5\linewidth}
\centering\large \includegraphics[width=.9\textwidth]{ComparativaLineas1}
\caption{Líneas después de unir}
\end{subfigure}
\caption{Ejemplo de una visualización del resultado intermedio de las funciones.}\label{fig:5.2}
\end{figure}


\item Desde el propio entorno puedes ejecutar no solo código estructurado en script sino también código estructurado en clases y llamadas a métodos es como un IDE pero con limitaciones. Como podemos observar en la imagen \ref{fig:5.3}

\begin{figure}[h]
\centering
\includegraphics[width=0.65\textwidth]{Ejecucion}
\caption{Ejemplo de una ejecución.}
\label{fig:5.3}
\end{figure}

\item Multitud de librerías y funciones que en entornos parecidos como matlab serian de pago y aquí al ser software libre el ejemplo anterior lo resume en una librería numpy \cite{Numpy}.
\end{itemize}

\section{Procesado imagen}
Para llegar a conseguir calcular las líneas que había pintadas en las imágenes tube que realizar una serie de pasos que vamos a resumir en tres etapas.
\subsection{Binarización}
Partiendo de una imagen que solo tenía líneas en rojo pintadas encima de las estrías producidas por el desgaste y lo demás de la imagen en escala de grises, lo primero fue leer la imagen a trabes de las funciones ya programadas en la librería de Scikit-Image(skimage) \cite{scik:skeleton}.\\
Una vez que tenemos la imagen guardada en el espacio de color RGB podemos empezar el procesado quedándonos con el canal Rojo.\\
Calculamos la distancia de cada pixel de la imagen al color rojo restando, uno menos el valor absoluto del pixel en el canal S (saturación) del espacio de color HSV,(restando el valor absoluto a la unidad conseguimos normalizar entre [0-1])y pasamos la imagen de distancias a blanco y negro y así tendremos un valor entre 0 y 256 en cada pixel correspondiente a la distancia al rojo cuanto mas alejado mas negro y los que sean rojos en blanco.\\
Para que la diferencia sea blanco o negro binarizamos la imagen con un valor umbral calculado como threshold otsu y así la imagen los valores de la distancia que sean mayores que el umbral pasaran a valer máximo y los que no consigan pasar el umbral serán los bordes blancos \ref{fig:5.4}.


\begin{figure}
\begin{subfigure}[c]{.5\linewidth}
\centering\large \includegraphics[width=.9\textwidth]{paso1Binariza}
\caption{Original}
\end{subfigure}%
\begin{subfigure}[c]{.5\linewidth}
\centering\large \includegraphics[width=.9\textwidth]{paso1DistanciaR}
\caption{Distancia al rojo}
\end{subfigure}
\begin{subfigure}[c]{.5\linewidth}
\centering\large \includegraphics[width=.9\textwidth]{paso1Binariza}
\caption{Imagen binarizada}
\end{subfigure}
\caption{Resumen visual binarización.}\label{fig:5.4}
\end{figure}


\subsection{Obtener segmentos}
Partimos de la imagen binarizada y lo primero es reducir el grosor de las líneas detectadas a un pixel eso lo conseguimos llamando a la función skeletonize que nos devuelve la imagen con las líneas de un pixel (así no acumulamos errores y es mas rápida la búsqueda de rectas).\\
Seguidamente llamamos a la función <<probabilistic hough line>> que nos va a encontrar segmentos que formaran las líneas el funcionamiento ha sido explicado en el apartado (conceptos teóricos). Como podemos observar en la figura \ref{fig:5.5}.


\begin{figure}
\begin{subfigure}[c]{.5\linewidth}
\centering\large \includegraphics[width=.9\textwidth]{paso2Binaria}
\caption{Binarizada}
\end{subfigure}%
\begin{subfigure}[c]{.5\linewidth}
\centering\large \includegraphics[width=.9\textwidth]{Paso2Skele}
\caption{líneas a 1 pixel}
\end{subfigure}
\begin{subfigure}[c]{.5\linewidth}
\centering\large \includegraphics[width=.9\textwidth]{Paso2Segmentos}
\caption{Imagen original con segmentos}
\end{subfigure}
\caption{Resumen visual Obtener Segmentos.}\label{fig:5.5}
\end{figure}

\subsection{Procesado de segmentos}
Llegados a este punto lo que tenemos son muchos segmentos que forman las líneas reales y tenemos que unirlos. Para ello vamos a usar la teoría de grafos añadiendo los segmentos a un grafo.\\
Para unir dos segmentos tiene que cumplirse que la distancia mínima entre sus extremos sea menor que un umbral y si pasa este punto comprobaremos que el angulo que forman entre ellas sea menor a otro umbral y si cumplen las dos condiciones añadiremos un camino al grafo desde la recta uno a la recta dos como se ve en la figura \ref{fig:5.6}.



\begin{figure}
\begin{subfigure}[b]{.5\linewidth}
\centering\large \includegraphics[width=.9\textwidth]{grafoLineasOri}
\caption{Imagen con segmentos en rojo.}
\end{subfigure}
\begin{subfigure}[b]{.5\linewidth}
\centering\large \includegraphics[width=.9\textwidth]{grafo}
\caption{Grafo de clusters de segmentos.}
\end{subfigure}
\caption{Resumen visual procesado de Segmentos.}\label{fig:5.6}
\end{figure}

\subsection{Recuperación de líneas}
Ahora lo que tenemos es un grafo con clusters ya que cada cluster se identifica con únicamente una recta y tendremos tantos como rectas.
Un problema de grafos es el problema de las k-componentes pero a nosotros solo nos interesan las 1-componentes del grafo ya que cada grupo de estos segmentos cercanos se corresponde con una recta real.
devolvemos la combinación de los segmentos mas relevantes de cada cluster y estos se convierten en nuestra buscada linea real \ref{fig:5.7}.
\begin{figure}[h]
\centering
\includegraphics[width=0.65\textwidth]{ResultanteDelGrafo}
\caption{Líneas obtenidas después de procesar el grafo}
\label{fig:5.7}
\end{figure}

\subsection{Resumen pasos}

\begin{itemize}
\item Binarizar la imagen para solo quedarnos con los objetos de interés
\item Obtener segmentos que forman trozos de las líneas.
\item Añadir caminos entre las líneas cercanas en un grafo.
\item Obtener los grupos de líneas próximas y devolver la recta que las una. 
\end{itemize}

\section{Interfaz}
\subsection{Primera versión}

Para el desarrollo de la interfaz gráfica pensé en una planificación espacial acorde con los elementos que esta contendría.
Un layout que seria muy adecuado podría ser un border layout pero como no existe en PyQt4 lo he tenido que simular gracias a apilar layouts de otros tipos.
Consiguiendo tener una botonería arriba del todo y dos columnas debajo claramente diferenciadas en la que en una estuviera la imagen que vamos a mostrar y en la otra las funcionalidades \ref{fig:5.8}.

\begin{figure}[h]
\centering
\includegraphics[width=0.65\textwidth]{disenoInter}
\caption{Diseño de la interfaz de usuario}
\label{fig:5.8}
\end{figure}

\subsection{Versión a evaluar}

En otro Sprint del proyecto lo que he usado han sido pestañas para así tener en la zona de funcionalidades los modos de trabajo de la aplicación claramente separados \ref{fig:5.9}.

\begin{itemize}
\item Detección líneas rojas.
\item Corrección y pintar líneas.
\item Automático.
\end{itemize}


\begin{figure}[h]
\centering
\includegraphics[width=0.65\textwidth]{disenoWi}
\caption{Diseño de la interfaz de usuario}
\label{fig:5.9}
\end{figure}
\subsubsection{Barra de herramientas}
Es una sección de la interfaz que nos va permitir de momento la carga de imágenes para su procesado \ref{fig:5.10}.
\begin{figure}[h]
\centering
\includegraphics[width=0.65\textwidth]{BarraHerramientas}
\caption{Barra de herramientas de la aplicación}
\label{fig:5.10}
\end{figure}

\subsubsection{Barra de herramientas de la imagen}
Es una sección de la interfaz que nos permitirá manipular la imagen para realizar estas acciones \ref{fig:5.11}.

\begin{itemize}
\item Volver al principio. 
\item Retroceder un paso.
\item Avanzar un paso.
\item Desplazar la imagen.
\item Aumentar la región que seleccionemos.
\item Configurar la región, bordes, etc.
\item Guardar la imagen con su escala.
\item configurar el tamaño y numeración de los ejes.
\item Coordenadas actuales del ratón.
\item Niveles de color de los canales R G B del espacio RGB.
\end{itemize}

\begin{figure}[h]
\centering
\includegraphics[width=0.95\textwidth]{toolbar}
\caption{Barra de herramientas de la imagen}\label{fig:5.11}
\end{figure}


\subsubsection{FigureCanvas de la imagen}
Es la región donde se muestra la imagen que va a ser ligeramente mas grande que la parte de las pestañas \ref{fig:5.12}.
\begin{figure}[h]
\centering
\includegraphics[width=0.65\textwidth]{FigureCanvas}
\caption{FigureCanvas de la imagen}
\label{fig:5.12}
\end{figure}


\subsubsection{Pestañas}
Sección de la imagen donde estarán implementadas las funcionalidades para visualizar las líneas que son detectadas por el algoritmo en la imagen \ref{fig:5.13}.
\begin{itemize}
\item El modo primero es la detección de los surcos en rojo en los dientes 
\item El modo segundo es la corrección/detección manual de las líneas por una persona y quedando reflejadas en la imagen.
\item El modo tercero es la ultima parte del proyecto que consistirá en hacer todo el proceso anterior de forma automática.
\end{itemize}

\begin{figure}[h]
\centering
\includegraphics[width=0.65\textwidth]{Pestanas}
\caption{Pestañas}
\label{fig:5.13}
\end{figure}

