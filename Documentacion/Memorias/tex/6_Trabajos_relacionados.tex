\capitulo{6}{Trabajos relacionados}

En esta sección, vamos a hablar de otros trabajos que están relacionados, tanto en la resolución del problema nuestro en particular, como resolver el problema de las detecciones de otro tipo de estrías o información.

\section{Artículo de Alejandro Pérez Pérez}
El articulo \cite{perez:perez}, es en que se basan en todos los artículos que hay hasta el momento, para resolver el problema de deducir la dieta a partir de las marcas dentales, lo mencionamos por su importancia.

En este artículo se habla de como a partir de las micro estrías bucales, observadas al microscopio. Podemos ver los distintos patrones que estas crean, dependiendo de la dieta que se llevara.
Una vez analizadas, medidas y clasificadas mediante las formulas que proponen podemos obtener un resultado que las caracteriza y encuadra en un plano donde figuran las distintas áreas que cada uno de las clasificaciones genera.

\section{Artículo de Rebeca García González}
El articulo \cite{Rebeca:garcia} de nuestro cliente en el que analiza un ejemplo para datar la dieta correspondiente.
La muestra que se evaluó en este artículo es la de <<El Mirón>>,  después de los análisis que proponen, Lalueza y Alejandro Pérez Pérez, no se corresponde con ninguna elipse de probabilidad del modelo propuesto.
Pero si que se clasifica por el segundo método que proponen en carnívoros.

\section{Trabajo fin de grado: Sergio Chico Carrancio}
Podemos acceder al proyecto de Sergio en : \url{https://github.com/Serux/perikymata}.
En este proyecto se centraron en detectar estrías, pero no eran las mismas que nosotros queremos detectar, sino unas estrías llamadas perykimatas, estas muestran el crecimiento del diente. Pero muchas de las funcionalidades y partes del programa, son parecidas a las nuestras, aunque solucionado de formas distintas.
