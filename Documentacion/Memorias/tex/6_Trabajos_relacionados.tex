\capitulo{6}{Trabajos relacionados}

Este apartado sería parecido a un estado del arte de una tesis o tesina. En un trabajo final grado no parece obligada su presencia, aunque se puede dejar a juicio del tutor el incluir un pequeño resumen comentado de los trabajos y proyectos ya realizados en el campo del proyecto en curso. 

\section{Paper de Alejandro Pérez Pérez:}
Podemos acceder a traves de este enlace \url{https://www.researchgate.net/profile/Alejandro_Perez-Perez/publication/14404290_Dietary_inferences_through_buccal_microwear_analysis_of_Middle_and_Late_Pleistocene_human_fossils/links/56f0075208ae3c65343663c3.pdf}.
Este articulo es en que se basan en todos los articulos que hay hasta el momento para resolver este problema, lo mencionamos por esta importancia.

\section{Paper de Rebeca García González}
Podemos acceder a traves de este enlace \url{http://ac.els-cdn.com/S0305440315001132/1-s2.0-S0305440315001132-main.pdf?_tid=d1043520-bc58-11e6-aa09-00000aacb35f&acdnat=1481100213_f3d349800071338152908049b1b3d86f}.
Este es el articulo de nuestro cliente en el que analiza un ejemplo para datar la dieta correspondiente.

\section{El Trabajo fin de grado : De Sergio Chico Carrancio}
Podemos acceder al proyecto de Sergio en : \url{https://github.com/Serux/perikymata}.
En este proyecto se centraron en detectar estrías, pero no eran las mismas que nosotros queremos detectar, sino unas estrías llamadas perykimatas, estas muestran el crecimiento del diente. Pero muchas de las funcionalidades y partes del programa, son parecidas a las nuestras, aunque solucionado de formas distintas.
