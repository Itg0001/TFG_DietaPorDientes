\capitulo{7}{Conclusiones y Líneas de trabajo futuras}

Todo proyecto debe incluir las conclusiones que se derivan de su desarrollo. Éstas pueden ser de diferente índole, dependiendo de la tipología del proyecto, pero normalmente van a estar presentes un conjunto de conclusiones relacionadas con los resultados del proyecto y un conjunto de conclusiones técnicas. 
Además, resulta muy útil realizar un informe crítico indicando cómo se puede mejorar el proyecto, o cómo se puede continuar trabajando en la línea del proyecto realizado. 



\section{Conclusiones:}
Para desarrollar este punto las vamos a ir separando por puntos dependiendo de la temática y de los campos a los que se refieran.
Pero van a ir cubriendo todas las conclusiones a las que hemos llegado, cubriendo todos los puntos mas relevantes que hemos realizado.

\begin{itemize}
\item Dinámica del proyecto: Para concluir con el proyecto, estoy muy satisfecho con la dinámica del proyecto, y con todos los aspectos que hemos tocado dentro del mismo.
Algunos aspectos de investigación han sido difíciles, no hay demasiada información respecto al tema, y somos los primeros en hacer una aplicación para resolver estos problemas, por lo que nos hemos encontrado aveces con impedimentos que hemos solucionado invirtiendo mas tiempo.

\item Transparencia al usuario: El reto mas grande ha sido conseguir realizar todas las funciones, con la mas sencilla transparencia para el usuario, ante todo hemos invertido parte del tiempo, no solo en resolver sino en pensar cual seria la forma mas sencilla, para que un usuario use el programa intuitivamente. 
\item Procesado y extracción de características: Los algoritmos para el procesado y la reconstrucción de las características que necesitábamos obtener, no eran directos. Como la variabilidad de tantos parámetros afectaba enormemente al resultado final, hemos tenido que ir testando muchos valores, para al final generalizarlos y conseguir un resultado bueno.

\item Uso del conocimiento adquirido durante la carrera: Durante el proceso de desarrollo hemos echo uso de muchos conceptos disjuntos que hemos dado durante la carrera y juntándolos para conseguir nuestras necesidades.

\item Lenguaje de desarrollo: El lenguaje de desarrollo, Python, es un plus para el proyecto ya que al investigar sobre este, hemos visto que últimamente en investigación y procedimientos matemáticos esta al alza, dispone de una gran cantidad de librerías con funciones muy actuales y de fácil aplicación. La programación es mas eficiente que en otros lenguajes.

\item Agilidad: Para completar este punto nosotros hemos tenido una curva de desarrollo muy estable y muy alta, ya que nos hemos centrado exclusivamente a este proyecto.
Por otra parte como agilidad en lo referido a metodología de desarrollo Scrum \cite{Scrum}, ha sido muy útil, porque en cada Sprint teníamos, una version con algunas capacidades añadidas, pero usable desde el primer momento.

\item Valor adquirido por mi: 
El principal motivo para la elección de este proyecto fue, el tema del mismo, es algo que si no se hace en colaboración con una universidad no es muy probable que surja, otro de los motivos fue el propio campo de la antropología, es algo que siempre me ha gustado y nunca tuve la oportunidad de conocer mas a fondo.
También como lenguajes de desarrollo elegí Python, porque en este lenguaje no sabia como hacer interfaces gráficas y como lo hemos dado en varias asignaturas quería terminarlo de completar sabiendo usar las interfaces ya que en contraposición en Java si que he visto y tocado en varias ocasiones el desarrollo de interfaces.

\item Los patrones de diseño:
Después de hacer este proyecto y las dificultades que han surgido, me he dado cuenta que las asignaturas de diseño debería de haberlas cursado, porque me han resuelto muchos problemas y me han ahorrado mucho tiempo.
Pero no viene de mal mencionar que al final las he usado y las he aprendido porque he querido completar mi formación.


\end{itemize}



\section{Líneas de trabajo futuras:}
