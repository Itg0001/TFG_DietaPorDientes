\capitulo{7}{Conclusiones y Líneas de trabajo futuras}


En este apartado, vamos a mencionar algunas conclusiones a las que hemos llegado después de realizar el proyecto, algunas partes sobre las que deberían recalcar mas adelante así como direcciones a tomar para futuros desarrolladores que continuasen el proyecto.



\section{Conclusiones}
Para desarrollar este apartado vamos a ir separando por puntos las conclusiones, dependiendo de la temática y de los campos que cubierto, en la realización del proyecto.
Pero vamos a ir mencionando todas las conclusiones a las que hemos llegado, cubriendo todos los puntos mas relevantes que nosotros hemos realizado.

\begin{itemize}
\item Dinámica del proyecto: Para concluir con el proyecto, estoy muy satisfecho con la dinámica del proyecto, y con todos los aspectos que hemos abordado dentro del mismo.
Algunos aspectos de investigación han sido difíciles, no hay demasiada información respecto al tema, y somos los primeros en hacer una aplicación para resolver estos problemas. Por ello nos hemos encontrado con impedimentos que hemos solucionado invirtiendo mas tiempo.

\item Transparencia al usuario: El reto mas grande ha sido conseguir realizar todas las funciones con la mayor transparencia para el usuario y que también fueran intuitivas. Ante todo hemos invertido parte del tiempo, no solo en resolver sino en pensar cual sería la forma mas sencilla, para que un usuario use el programa intuitivamente. 

\item Procesado y extracción de características: Los algoritmos para el procesado y la reconstrucción de las características que necesitábamos obtener, no eran directos. Como la variabilidad de tantos parámetros afectaba enormemente al resultado final, hemos tenido que ir testando muchos valores, para al final ajustarlos a valores que obtuvieran un resultado adecuado.

\item Uso del conocimiento adquirido durante la carrera: Durante el proceso de desarrollo hemos usado muchos conceptos disjuntos, que hemos aprendido durante la carrera y juntándolos para conseguir nuestras necesidades.

\item Lenguaje de desarrollo: El lenguaje de desarrollo, Python, es un plus para el proyecto, ya que al investigar sobre este hemos visto que últimamente, en investigación y procedimientos matemáticos esta al alza. Dispone de una gran cantidad de librerías con funciones muy actuales y de fácil aplicación. La programación es mas eficiente que en otros lenguajes.

\item Agilidad: Para completar este punto yo he tenido una curva de desarrollo muy estable y muy alta, ya que me he centrado exclusivamente a este proyecto.
Por otra parte como agilidad en lo referido a metodología de desarrollo Scrum \cite{Scrum}, ha sido muy útil, porque en cada Sprint teníamos, una version con algunas capacidades añadidas, pero usable desde el primer momento.

\item Valor adquirido para mí: 
El principal motivo para la elección de este proyecto fue, el tema del mismo, es algo que si no se hace en colaboración con una Universidad no es muy probable que surja. Otro de los motivos fue el propio campo de la antropología, es algo que siempre me ha gustado y nunca tuve la oportunidad de conocer mas a fondo.
También como lenguajes de desarrollo elegí Python, porque en este lenguaje no sabía como hacer interfaces gráficas y como lo hemos dado en varias asignaturas quería terminarlo de completar sabiendo usar las interfaces, ya que en contraposición con Java si que he visto y tocado en varias ocasiones el desarrollo de interfaces.

\item Los patrones de diseño:
Después de hacer este proyecto y las dificultades que han surgido, me he dado cuenta que las asignaturas de diseño debería de haberlas cursado, porque me han resuelto muchos problemas y me han ahorrado mucho tiempo.
Pero no viene de mal mencionar que al final las he usado y las he aprendido porque he querido completar mi formación.

\end{itemize}



\section{Líneas de trabajo futuras}
Como partes de mejora futuras vamos enumerar varios aspectos del proyecto.

\begin{itemize}
	\item Detección automática de bordes: 
Podríamos proponer el incluir un algoritmo mas especializado en la detección de bordes, ya que las limitaciones que nos hemos encontrado hacen que este sea el puto mas flojo del proyecto. Como propuesta, podría ser la creación desde cero de un algoritmo que detectase los bordes de una manera mas precisa.

	\item Unión de segmentos: 
La transformada de Hough para la detección de las lineas ha dado muy buenos resultados, pero creo que se podría mejorar con este algoritmo del siguiente modo:

	\begin{itemize}
		\item Binarizar como hasta ahora.
		\item Esqueletonizar la imagen como hasta ahora
		\item Detectar las lineas mas grandes en la imagen y borrarlas de la máscara.
		\item Detectas las lineas bastante mas pequeñas pero aun no la mas cortas del todo y borrarlas de la máscara.
		\item Detectar las lineas cortas en la imagen y borrarlas de la máscara
		\item Filtrar las lineas que en la máscara no contengan mas de un 70\% de blanco. (Así quitamos falsos positivos).
	\end{itemize}
	\item Añadir mas tipos de estrías que podríamos detectar como modos auxiliares para determinar que características queremos extraer de la imagen. Es decir, acoplar en la aplicación la resolución de un nuevo problema de detección.
	
\end{itemize}