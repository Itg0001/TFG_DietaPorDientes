\apendice{Especificación de Requisitos}

\section{Introducción}



Este anexo contendrá los objetivos de la aplicación y los requisitos que va a tener esta.
Consistirán en breves descripciones de todos los objetivos principales, como se subdividen en objetivos mas cortos para conseguir realizarlos y las funcionalidades implementadas.

Requisitos: Contendrá el que consiste el proyecto y las definiciones de los requisitos y objetivos.


\section{Objetivos generales}
Los objetivos que buscamos con este proyecto son:

\begin{itemize}
\item Construcción del prototipo de procesado de las imágenes:
Con este prototipo la funcionalidad buscada era, hacer las pruebas sobre el código, que mas adelante introduciríamos en la aplicación real, pero así podríamos ajustar los parámetros para que funcionase bien. También de esta forma estaríamos seguros que el proyecto es realizable.

\item Construcción de la aplicación: 
Una vez que tengamos el prototipo funcionando, tendríamos que construir una aplicación para ejecutar de una forma mas profesional que simplemente desde un notebook, esto es algo muy avanzado para un usuario sin experiencia, así facilitar la ejecución sin necesidad de depender del notebook.

\item  Construcción del prototipo para el modo automático:
Como hemos realizado el trabajo en tiempo, hemos optado por acoplar un modo de detección de bordes automatizado, para ver si eramos capaces de detectar los segmentos que detectaría un humano.
El problema de este modo es que algunos segmentos no son casi perceptibles por un humano por lo tanto los que estén mas marcados son los que podremos detectar.

\item Añadir el modo automático a la aplicación: 
Como hemos echo anteriormente con el modo de detección de las lineas pintadas. Esta vez también lo acoplaremos a nuestra aplicación, facilitando la ejecución y aplicación de este modo sobre imágenes que no estuvieran pintadas.
\end{itemize}



\section{Catalogo de requisitos}
El proyecto consiste en:
\begin{itemize}
	\item Construcción del prototipo de procesado de las imágenes.
		\begin{itemize}
			\item Lectura de imagen.
			\item Binarización para detectar las pintadas.
			\item Procesado de la imagen binaria.
			\item Extracción de características.
			\item Procesado de las características.
		\end{itemize}
	\item Construcción de la aplicación.
		\begin{itemize}
			\item Lectura de imágenes y cargar en el panel.
			\item Construcción del panel de pestañas
			\item Construcción del modo de automático para lineas pintadas.
			\item Construcción del modo manual para corregir los errores o editar las lineas que hemos pintado.
			\item Construcción de la pestaña para el modo completamente automático.			
		\end{itemize}
	\item Construcción del prototipo para el modo automático.
		\begin{itemize}
			\item Lectura de la imagen.
			\item Equalizacion de la imagen para distribuir el histograma.
			\item Binarización para extraer bordes.
			\item Procesado de la imagen binaria para limpiar de ruido.
			\item Calculo de las características de la imagen.
			\item Procesado de características.		
		\end{itemize}
\end{itemize}


\section{Especificación de requisitos}


