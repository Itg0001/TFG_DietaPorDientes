\apendice{Especificación de Requisitos}

\section{Introducción}
Como introducción y para no repetirnos voy a explicar de forma mas general y sin entrar en detalles que contiene el proyecto.


El anexo contiene:

\begin{itemize}
\item Plan de proyecto: Consiste en la planificación temporal que hemos seguido adjuntando los gráficos de todas las semanas de trabajo.
\item Requisitos: Contendrá el que consiste el proyecto y las definiciones de los requisitos.
\item Diseño: Va a contener los diagramas de clases y la distribución en paquetes que hemos elegido. También contendrá que tiene cada clase.
\item Manual programador:
\item Manual del usuario: Este contendrá un tutorial de como usar la aplicación de forma correcta para la extracción de características de las imágenes.
\item Investigación: Este apartado contendrá que hemos probado y como hemos investigado para el desarrollo del modo automático de la aplicación.
\end{itemize}

\section{Objetivos generales}
En este apartado lo que vamos a enumerar son los objetivos que nos hemos propuesto y hemos cumplido de todo el proyecto.

El proyecto consiste en:
\begin{itemize}
	\item Construcción del prototipo de procesado de las imágenes.
		\begin{itemize}
			\item Lectura de imagenen.
			\item Binarización para detectar las pintadas.
			\item Procesado de la imagen binaria.
			\item Extracción de características.
			\item Procesado de las características.
		\end{itemize}
	\item Construcción de la aplicación.
		\begin{itemize}
			\item Lectura de imágenes y cargar en el panel.
			\item Construcción del panel de pestañas
			\item Construcción del modo de automático para lineas pintadas.
			\item Construcción del modo manual para corregir los errores o editar las lineas que hemos pintado.
			\item Construcción de la pestaña para el modo completamente automático.			
		\end{itemize}
	\item Construcción del prototipo para el modo automático.
		\begin{itemize}
			\item Lectura de la imagen.
			\item Equalizacion de la imagen para distribuir el histograma.
			\item Binarización para extraer bordes.
			\item Procesado de la imagen binaria para limpiar de ruido.
			\item Calculo de las características de la imagen.
			\item Procesado de características.		
		\end{itemize}
\end{itemize}




\section{Catalogo de requisitos}
Este apartado se refiere a las tareas que hemos realizado en cada uno de los respectivos Sprints.

Sprint 0
\begin{itemize}
	\item Probar LaTex 
	\item Probar gestores de tarea: 
	\begin{itemize}
	\item Trello 
	\item Zenhub 
	\item Version One
	\end{itemize}
	\item Gestores de versiones 
	\begin{itemize}
	\item GitHub 
	\item Bitbucket 
	\end{itemize}
	\item Examinar el problema, evaluar el notebook y las posibilidades de las librerías \\
	\item Echar un vistazo al artículo 
\end{itemize}



Sprint 1 
\begin{itemize}
\item Mejora de la detección de las líneas que quedas solapadas. 
\item Analizar herramientas de interfaces gráficas y comparativa.
	\begin{itemize}
	\item PyQt4. 
	\item WxWidget.
	\end{itemize}
\item Prototipado inicial de la herramienta y documentar el prototipo.
\end{itemize}



Sprint 2 
\begin{itemize}
\item Documentación. 
\begin{itemize}
	\item Aspectos relevantes. 
	\item Técnicas y herramientas.
	\item Planificación temporal.
	\end{itemize}
\item Listas en la interfaz gráfica (Usar tablas para mostrar las rectas que hemos añadido manualmente).
\item Cargar imágenes con el file chooser.
\end{itemize}

Sprint 3 
\begin{itemize}
	\item Acabar la GUI.
		\begin{itemize}
			\item Mostrar todas las líneas en la tabla.
			\item Poder borrar la línea seleccionada.
		\end{itemize} 
	\item Informe.
		\begin{itemize}
			\item Calcular estadísticas.
			\item Mirar documentación Python.
		\end{itemize}
	\item Pasar código de PyQt4 a PyQt5.
\end{itemize}


Sprint 4
\begin{itemize}
	\item Diseño de la aplicación.
		\begin{itemize}
			\item Diagrama de clases.
			\item Diagrama de paquetes.
		\end{itemize}
		
	\item Implementación del código.
		
	\item Corrección de las memorias.
	
	\item Herramienta SonarQube.
	\begin{itemize}
		\item Ejecutar la aplicación.
		\item Corregir las horas de débito.
	\end{itemize}
\end{itemize}



Sprint 5
\begin{itemize}
	\item Aplicar patrones de diseño
	\begin{itemize}
		\item Fachada.
		\item Mediador.
		\item Comando.
	\end{itemize}
	\item Documentar sobre los patrones usados.
	\item XML mirar documentación sobre ello.
\end{itemize}



Sprint 6 
\begin{itemize}
\item Completar menús de la GUI.
	\begin{itemize}
		\item Guardar.
		\item Cargar.
		\item Acerca de.
		\item Ayuda.
	\end{itemize}
\item Completar documentación de los fuentes.
	\begin{itemize}
		\item Paquete código.
		\item Paquete gui.
	\end{itemize}
\item Pruebas unitarias.
	\begin{itemize}
		\item Teses paquete código.
		\item Teses paquete gui.
	\end{itemize}
\end{itemize}




\section{Especificación de requisitos}


