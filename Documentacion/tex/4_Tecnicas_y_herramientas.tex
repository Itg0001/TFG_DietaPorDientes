\capitulo{4}{Técnicas y herramientas}

%\maketitle
\section{Interfaz gráfica de usuario:}
\subsection{Tkinter}
Es una librería que proporciona el diseño y visualización de interfaces de usuario en Python esta a su vez esta basada en librerías de TK/TCL que están incluidas por en la propia instalación.
\\

\textbf{Ventajas:}

\begin{itemize}
\item Es fácil de usar y es recomendable para el aprendizaje del lenguaje.

\item Viene preinstalado con la distribución por lo que su uso es inmediato

\item Al servir para aprendices y venir preinstalado podemos encontrar multitud de tutoriales y de documentación sobre ello.
\end{itemize}

\textbf{Desventajas:}

\begin{itemize}
\item Pocos elementos gráficos., escaso control de las ventanas y bastante lento.
\end{itemize}

\subsection{WxPython}
Es una librería basada en otra importante que veremos mas adelante , también es multiplataforma y esta programada en C/C++, es mas nueva que Tkinter.
\\

\textbf{Ventajas:}

\begin{itemize}
\item Es más difícil de usar que Tkinter pero aun así hay mucha documentación sobre ella.

\item Dispone de gran cantidad de elementos gráficos por lo que es bastante potente aunque con alguna limitación respecto a otras.

\item Permite hacer una barrera o separación entre el código Python y lo que es la interfaz.
\item Cuenta con una gran comunidad de gente que lo usa y postea ejemplos y tutoriales.
\end{itemize}

\textbf{Desventajas:}

\begin{itemize}
\item La principal desventaja es que se actualizan las versiones mucho y para mantener una aplicación durante largo tiempo perdemos tiempo de.
\item Es más complejo de usar que el anterior.
\end{itemize}

\subsection{Pyqt}
Es más difícil de usar que WxPython y WxWidget pero da más control sobre los elementos gráficos y muchas librerías se basan en ello por lo que se puede encontrar bastante información sobre esta librería.
\\

\textbf{Ventajas:}

\begin{itemize}
\item Al ser tan usada si instalamos Python desde Anaconda que es un cojunto de librerías y aplicaciones de Python ya tendríamos pyqt4 por defecto instalado.
\end{itemize}

\textbf{Desventajas:}

\begin{itemize}
\item Es mas difícil de entender y comprender que los anteriores pero queda mas limpias las interfaces.
\item Si somos puristas e instalamos Python solo sin ide ni nada no vendría instalado pero si usamos Anaconda si que vendría instalado
\end{itemize}

\subsection{WxWidgets:}
Como hemos mencionado anteriormente es muy parecido a WxPython por lo que no vamos a entrar en detalle también está programado en C/C++ y es multiplataforma da un aspecto de comportamiento nativo.
\\


\section{Plantillas}
Las plantillas sirven para generar código con la sustitución de los valores dentro de sus variables y así tener siempre un código con los valores en la ejecución.\\
Nosotros las vamos a usar en combinación con LaTex para generar el pdf con el informe de las estadísticas de la ejecución del algoritmo que detecta las líneas.\\
Como podemos observar hay muchos frameworks para usar las plantillas pero hemos elegido comparar varios usados otros años por compañeros.
\subsection{Mustache}
Pystache es una implementación de Mustache en Python para el diseño de plantillas, está inspirado en Ctemplate y et.
En estos tipos de lenguaje no se puede aplicar lógica de aplicación simplemente son para una presentación más fluida.

\textbf{Ventajas:}

\begin{itemize}
\item Tiene documentación online.
\item Está disponible para una gran variedad de lenguajes de programación.
\item Si necesitásemos hacer documentos xml seria la herramienta perfecta.
\end{itemize}

\textbf{Desventajas:}
\begin{itemize}
\item Carece de ejemplos o tutoriales de cómo usarlo con LaTex en su versión de Python.
\end{itemize}


\subsection{Jinja2}

El nombre de la librería viene del templo Japonés Jinja.\\
Es una librería para escribir plantillas, para Python, funciona con las últimas versiones de Python, también es una de las más usadas librerías está inspirada en Django.


\textbf{Ventajas:}
\begin{itemize}
\item Viene preinstalado con Anaconda ya que es uno de los más utilizados.
\item Fácil de usar y de pasar los parámetros.
\item Fue creada para Python2 pero también funciona en Python3
\end{itemize}
\textbf{Desventajas:}
\begin{itemize}
\item No esta implementado más que para Python.
\end{itemize}

\subsection{Comparación de las herramientas}
Esta comparación son dos columnas\ref{tb:1} sacada de la tabla comparativa de herramientas obtenida en wikipedia.


\begin{table}[]
\centering
\caption{Tabla comparativa de herramientas}
\label{tb:1}
\resizebox{\textwidth}{!}{
\begin{tabular}{|l|l|l|l|l|l|l|l|l|l|l|l|l|}
\hline
Biblioteca & Lenguaje & Licencia & Variables & Funciones & Include & \begin{tabular}[c]{@{}l@{}}Inclusiones\\ condicionales\end{tabular} & Bucles & Evaluacion & Asignaciones & Errores y excepciones & \begin{tabular}[c]{@{}l@{}}Plantillas\\ naturales\end{tabular} & Herencia \\ \hline
Jinja2     & Python   & BSD      & SI        & SI        & SI      & SI                                                                  & SI     & SI         & SI           & SI                    & SI                                                             & SI       \\ \hline
Mustache   & +30      & MIT      & SI        & SI        & SI      & SI                                                                  & SI     & NO         & NO           & SI                    & SI                                                             & NO       \\ \hline
\end{tabular}
}
\end{table}


