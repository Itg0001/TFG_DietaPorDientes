\capitulo{4}{Técnicas y herramientas}

%\maketitle
\section{Interfaz gráfica de usuario:}
\subsection{Tkinter}
Es una librería que proporciona el diseño y visualización de interfaces de usuario en Python esta a su vez esta basada en librerías de TK/TCL que están incluidas por en la propia instalación.
\\

\textbf{Ventajas:}

\begin{itemize}
\item Es fácil de usar y es recomendable para el aprendizaje del lenguaje.

\item Viene preinstalado con la distribución por lo que su uso es inmediato

\item Al servir para aprendices y venir preinstalado podemos encontrar multitud de tutoriales y de documentación sobre ello.
\end{itemize}

\textbf{Desventajas:}

\begin{itemize}
\item Pocos elementos gráficos., escaso control de las ventanas y bastante lento.
\end{itemize}

\subsection{WxPython}
Es una librería basada en otra importante que veremos mas adelante , también es multiplataforma y esta programada en C/C++, es mas nueva que Tkinter.
\\

\textbf{Ventajas:}

\begin{itemize}
\item Es más difícil de usar que Tkinter pero aun así hay mucha documentación sobre ella.

\item Dispone de gran cantidad de elementos gráficos por lo que es bastante potente aunque con alguna limitación respecto a otras.

\item Permite hacer una barrera o separación entre el código Python y lo que es la interfaz.
\item Cuenta con una gran comunidad de gente que lo usa y postea ejemplos y tutoriales.
\end{itemize}

\textbf{Desventajas:}

\begin{itemize}
\item La principal desventaja es que se actualizan las versiones mucho y para mantener una aplicación durante largo tiempo perdemos tiempo de.
\item Es más complejo de usar que el anterior.
\end{itemize}

\subsection{Pyqt}
Es más difícil de usar que Wxpython y wxwidget pero da más control sobre los elementos gráficos y muchas librerías se basan en ello por lo que se puede encontrar bastante información sobre esta librería.
\\

\textbf{Ventajas:}

\begin{itemize}
\item Al ser tan usada si instalamos Python desde anaconda que es un cojunto de librerías y aplicaciones de Python ya tendríamos pyqt4 por defecto instalado.

\end{itemize}

\textbf{Desventajas:}

\begin{itemize}
\item Es mas difícil de entender y comprender que los anteriores pero queda mas limpias las interfaces.
\item Si somos puristas e instalamos Python solo sin ide ni nada no vendría instalado pero si usamos anaconda si que vendría instalado
\end{itemize}

\subsection{WxWidgets:}
Como hemos mencionado anteriormente es muy parecido a wxpython por lo que no vamos a entrar en detalle también está programado en C/C++ y es multiplataforma da un aspecto de comportamiento nativo.
\\








